\section{Conclusion}
\label{sec:diffusion_conclusion}

The diffusion framework described in \secref{sec:disney_motion_diffusion} proves to represent an exciting research direction. The baseline model shows good performance on a variety of tasks, as seen in \secref{sec:baseline_evaluation}. It is capable of inpainting the legs of a motion sequence, of plausibly filling in randomly occluded states, of inbetweening motion sequences, and of generating new motion sequences from scratch. The model also shows promise for other tasks such as pose autocompletion. This demonstrates that the diffusion framework can learn a general-purpose motion prior model without task-specific fine-tuning or architecture design. We are positive that future research directions in which the specific tasks to solve are more precisely described, and where more fine-grained architecture, data and training choices could be made, would prove fruitful avenues with lots of opportunity. The implementation of the diffusion framework paves the way for further investigation into its uses in the context of general animation tasks and proves itself as another useful tool that can be drawn upon in the future.