\section{Conclusion}

The DMD diffusion framework described in \secref{sec:disney_motion_diffusion} proves to represent an exciting research direction. The baseline model shows good performance on a variety of tasks, as seen in \secref{sec:baseline_evaluation}. It is capable of inpainting the legs of a motion sequence, of plausible filling in randomly occluded state, of inbetweening motion sequences, and of generating new motion sequences from scratch. The model also shows promise for other tasks such as pose autocompletion. This demonstrates that the diffusion framework is able to learn a general purpose motion prior model without task specific fine tuning or architeture design. We are positive that future research directions in which the specific task to solve is more precisely described, and where more fine grained architecture, data and training choices could be made, would prove fruitful avenues with lots of oppertunity. The implementation of the diffusion framework within the Disney Research|Studios codebase paves the way for further investigation into it's uses in the context of general animation tasks, and proves itself as another useful tool that can be drawn upon in the future.