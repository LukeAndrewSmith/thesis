The authors of HuMoR \cite{humor} present a novel approach for learning and using a plausible motion prior. They train a conditional VAE that learns a distribution over latent transitions, in a canonical reference frame, between $\textit{states}$ that consist of a root translation, 3D joint positions, joint angles, and the respective velocities. They most notably use this model as a prior in a 'test time optimisation', which generates plausible sequence motions optimising for an initial state and a sequence of transitions starting from frame by frame estimates (2D/3D joints or points clouds). This optimisation includes, alongside others, a motion prior term based upon the conditional distribution $p(z_t|x_{t-1})$ that encourages plausible motion for the learned sequence. Note that the CVAE decoder also predicts foot contacts alongside change in state, which are used as regularisers during their main use case of 'test time optimisation'. The test time optimisation can operate on many modalities, 2D/3D joints, point clouds, etc., as the optimisation loss contains a Data Term $\epsilon_{data}$ that can be tailored to the modality as the HuMoR state is information rich, containing 3D joints (hence can fit to 2D joints through projection or directly to 3D) and can parametrise the SMPL model (hence the SMPL mesh can be correlated to point clouds). The initialisation for the test time optimisation is based upon VPoser \TODO{Complete}. 

The performance of HuMoR as described in the paper \cite{humor}, alongside it's use in a problem that directly matches our own, lead us to evaluate and investigate the model \secref{sec:humor_investigation}, and subsequently try to extend and improve \ref{sec:humor_improvement} upon it once the limitations became clear.

