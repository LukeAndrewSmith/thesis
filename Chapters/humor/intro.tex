The authors of HuMoR \cite{humor} present a novel approach for learning and using a plausible motion prior. They train a conditional VAE that learns a distribution over latent transitions from a given state, and can decode a sample from this distribution to a change of state. They notably use this model as a prior in a 'test time optimisation' (TestOps), which generates plausible sequence motions optimising for an initial state and a sequence of transitions starting from frame by frame estimates. The TestOps can operate on many modalities, 2D/3D joints, point clouds, etc., as the optimisation loss contains a Data Term $\epsilon_{data}$ that can be tailored to the modality as the HuMoR state is information rich, containing 3D joints (hence can fit to 2D joints through projection or directly to 3D) and can parametrise the SMPL model (hence the SMPL mesh can be correlated to point clouds). 

The performance of HuMoR as described in the paper \cite{humor}, alongside it's use in a problem that directly matches our own (TestOps operating on RGB video) lead us to evaluate and investigate the model \secref{sec:humor_investigation}, and subsequently try to extend and improve \ref{sec:humor_improvement} upon it once the limitations made themselves known.


