We embarked upon the work presented in this thesis to explore motion modeling in the context of a project at Disney Research|Studios aiming to capture motion data direct from RGB videos. To begin with, we explored the relevant literature in \chpref{chpt:related_work}, gaining a broader understanding of the field of motion modeling, focusing specifically on the common Autoencoder and VAE architectures, and on the promising field of diffusion models. Next, we selected the HuMoR method \cite{humor} as a base for an investigation, with the view that the models strong performance and use for rectifying motion sequences obtained from 2d pose estimates, a problem that matched exactly the Disney Research|Studious pipeline, could lead to an important improvement of the existing pipeline. We investigated the model in \chpref{chpt:humor}, finding that the model provided promising result however at a huge computational cost. We set out to improve this speed issue in \secref{sec:humor_improvement}, presenting a novel strategy for optimisation that broke the autoregressive nature of the HuMoR method at the heart of the performance issue. Considerable effort was made to obtain comparable results, however the conclusion was reached that the results were not nearing the required quality and that the desired speedup seemed tricky to achieve. With the lessons of this experience on hand we formulated a new plan, choosing to investigate a sequence-level model, rather than the local pose-to-pose model, due to the potentially large speedup, and moved to the novel diffusion framework for which a promising literature is emerging. We explored in detail the diffusion framework in \chpref{chpt:diffusion} and formulated a baseline architecture upon which to base our investigations. We evaluated several models on a wide range of different tasks in \secref{sec:diffusion_experiments}, finding that the model can successfully replace missing legs and random missing joints, that it shows promise for motion inbetweening, that it generates diverse plausible motion, that it can be used for other tasks such as pose autocompletion, that the diffusion denoising procedure is flexible and can be sped up, and that the inpainting procedure can be modified to improve performance. Finally, we discussed the many exciting future research directions in \secref{sec:diffusion_future_work}.