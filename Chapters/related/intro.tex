The study of synthesizing human motion has a long history motivated in no small part by the desire to create realistic and captivating media in the gaming and film industries. Early adaptive methods rely on motion matching \cite{early_motion_matching} \cite{clavet_motion_matching} in which interpolation is performed between similar motions from a database of captured motion. This, however, has a high memory footprint, though efforts have been made to introduce learned aspects to compress the data into faster-to-query networks, such as \cite{holden_motion_matching}.

More recent branches of motion modeling commonly base themselves upon the use of machine learning techniques, notably deep learning, to learn a prior over plausible motion. This is a more general approach that can be applied to a wider range of tasks, and shows promise in overcoming some of the issues of motion matching, as such systems can learn to better generalise to out-of-distribution motion sequences.

Within the area of deep learning, many techniques have been investigated, temporal convolutions \cite{temporal_convolutions}, recurrent models \cite{recurrent_harvey_2020}, and reinforcement learning \cite{rl_cho} are but a few examples.
